\begin{document}
Одной из наиболее актуальных проблем современной прикладной физики является
получение источников когерентного излучения терагерцового (THz) диапазона. Такие
источники могли бы использоваться во множестве медицинских приложений, ввиду достаточно 
малого поглащения этого излучения тканями человека, что позволило бы разработать 
новые неинвазивные методы диагностики онкологии и иных заболеваний; другим возможным 
применением может стать спектроскопия сложных органических соединений, поскольку 
они имеют вращательные и колебательные степени свободы, имеющие соответствующие частоты 
лежащие именно в терагерцовом диапазоне.

В настоящее время имеется несколько способов генерации подобного излучения, которые, однако, 
имеют много недостатков. Одним из классов таких приборов являются квантово-каскадные лазеры (QCL).
Они демоснтрируют превосходные характеристики (высокий КПД, высокий уровень когерентности)
в диапазоне 1-5 THz и выше 15 THz. Однако большая часть таких лазеров создается на основе 
полупроводников типа A3B5 (GaAs, PbSb или InP), которые имеют высокое поглощение на оптических 
фононах в диапазоне 5-15 THz.

Альтернативой QCL являются лазеры на основе межзонных переходах в узкозонных полупроводниковых структурах.
Такие лазеры намного проще в изготовлении, а также могут излучать в диапазоне 5-15 THz. Однако ограничивающим
фактором является процесс безызлучательной Оже-рекомбинации. При этом традиционна ожидается, что этот процесс
рекомбинации будет весьма существенным.

Оже-рекомбинация представляет собой безызлучательный трёхчастичный процесс. По типу носителей заряда, участвующих
в процессе он делится на CCV и VVC процессы (процесс с участием двух электронов и дырки и процесс с двумя дырками
и электронм соответственно). В ходе этого процесса пара носителей с противоположным зарядом взаимоуничтожается и
передаёт энергию и импульс третьему. В силу выполнения законов сохранения этот процесс является пороговым.
Поэтому температура сильнг влияет на темп таких процессов, а значит и на эффективность лазеров.

Повлиять на это можно, варьируя материалы/структуры и изменяя тем самым дисперсионные соотношения в них. В частности,
существуют законы дисперсии, в которых такие процессы принципиально запрещены законами сохраниения (к примеру Дираковский 
или графеноподобный закон дисперсии), а возможны структуры, в которых энергетический порог таких эффектов стремится к нулю.

Спектральный диапазон 5 - 15 ТГц к настоящему моменту частично 
перекрыт лишь диодными лазерами на основе халькогенидов свинца-олова, которые обеспечивают длины 
волн излучения вплоть до 46.5 мкм. Фактор, который снижает эффективность оже-рекомбинации в PbSnSe(Te)
- симметрия между законами дисперсии носителей в зоне проводимости и в валентной зоне. Однако их рабочие характеристики 
ограничены технологией роста: существуют труднопреодолимые проблемы в реализации квантовых ям (КЯ) для твердых растворов 
PbSnSe(Te) и остаточная концентрация носителей остается на высоком уровне $10^17 \text{cm}^{-3}$.

Существуют альтернативные полупроводниковые системы, которые позволяют
приблизиться к «графеноподобному» закону дисперсии, но сохранить конечную ширину запрещенной зоны. 
Как было показано в многочисленных работах, одна из таких систем -- гетероструктуры с КЯ на основе Hg(Cd)Te/CdHgTe. 
В отличие от графена, в структурах на основе HgCdTe (КРТ) с КЯ можно перестраивать ширину запрещенной зоны путем изменения ширины КЯ и 
содержания Cd в ней. Современная молекулярно-лучевая эпитаксия (МЛЭ) обеспечивает высокое качество эпитаксиальных пленок КРТ не только на подложках CdZnTe, 
но и на «альтернативных» подложках GaAs. Высокое качество эпитаксиальных структур HgCdTe, выращенных на GaAs подложках, было подтверждено в ходе исследований 
фотопроводимости (ФП) и фотолюминесценции (ФЛ) в среднем и дальнем инфракрасном диапазонах ($λ$ = 15-30 $\mu m$). Было получено 
стимулированное излучение (СИ) в КРТ структурах с КЯ на длине волны $λ$ = 19.5 $\mu m$, в то время как ранее лазерная генерация в HgCdTe была 
продемонстрирована лишь в коротковолновой области среднего инфракрасного диапазона спектра (на длинах волн 2 - 5 $\mu m$). 
Для структур, рассчитанных на генерацию длинноволнового излучения, требуется рост толстых эпитаксиальных слоев (общей толщиной до 20 мкм) для реализации диэлектрического волновода. 
Целью работы настоящей работы было исследование длинноволнового СИ из подобных структур и выявление факторов, определяющие наблюдаемые характеристики СИ, и возможных путей подавления 
безызлучательной рекомбинации в таких структурах. В работе было продемонстрировано, что генерация СИ в таких структурах возможна. Тем не менее, профиль КЯ в волноводных 
структурах отличается от прямоугольного, что необходимо учитывать при анализе экспериментальных данных.
\newpage
\end{document}