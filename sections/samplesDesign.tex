\begin{document}
    \begin{center}
        \subsection{Дизайн гетероструктур}
    \end{center}

    Во многом дизайн рассматриваемых гетероструткур обусловлен технологическими особенностями их роста.
    Все образцы были получены посредством молекулярно "= лучевой эпитаксии в ИФП СО РАН.
    Обязательным условием возможности создания таких структур является буфер между подложкой 
    и участком с квантовыми ямами, обеспечивающий согласование приодов решётки кристалла, а также нивелирующий
    до некоторой степени разность в коэффициентах термического расширения образцов. Также буферный слой играет 
    роль нижней частью волновода, рассчитанного на длину волны, соответствующую нижнему возбужденному состоянию квантовых ям.

    Сам участок с квантовыми ямами растится в автоматическом режиме, что позволяет с высокой точностью выдерживать
    требуемый профиль прямоугольных квантовых ям, характерная толщина которых составляет 3-10 nm при ошибке порядка 0.1 nm.
    Однако подобный режим роста не позволяет обеспечивать требуемую точность концентрации, которая может нарушаться ввиду оседания
    части напыляемого материала на стенках камеры.

    Также следует отметить влияение количества квантовых ям на возможный уровень фотолюминисценции. В данном случае мы можем расматривать
    структуры как лазер с квазидвухуровневой средой. Такое рассмотрение возможно в силу сравнения времён накачки и теплового рассеяния:
    $\tau_p \gg \tau_\varepsilon$. Как известно в таких средах огромный вклад вносят процессы перепоглощения испускаемых фотонов. При таком
    подходе видно, что КПД устройства будет снижаться при большом количестве КЯ, что обусловлено увеличением эффективного времени жизни инверсии
    насселенностей и некоторым количеством потерь за счет паразитных безызлучательных механизмов реклмбинации (Оже и ШРХ). Однако в отдельности
    каждая квантовая яма имеет крайне низкий уровень поглощения излучения накачки. Из этого видно наличие некоего
    оптимального числа квантовых ям. В текущих образцах количество КЯ варьируется в диапазон 5-10 штук.

    Вторым ключевым элементом конструкции подобных гетероструктур является волновод, который с одной стороны позволяет создать стоячую волну, которая
    при правильном подборе параметров позволяет локализовать пучности ЭМ поля в точках, имеющих и наибольший квадрат модуля огибающей волновой функции электронов, 
    что позволяет существенно увеличить матричный элемент перехода с нижнего энергетического урованя на первый возбужденный.

    Однако в данном случае нас больше всего интересует связь между возникновением побочных максимумов и некоторым количеством 
    кадмия, который оказывается напыленным внутри квантовых ям. Очевидно, что использование более чистых в этом смысле структур
    может предотвратить возникновение канала ОЖЕ рекомбинации типа EEH через них. Однако кроме того методом численной оптимизации параметров 
    подобных структур могут быть получены параметры, при которых порог Оже рекомбинации в теории будет являться бесконечным (в частности можно добиться почти 
    гиперболического закона дисперсии). Однако на текущий момент подобные структуры не могут быть выращены в силу высоких требований к чистоте исполнения и 
    прецизионного выдерживания толщины КЯ. Более того в таких случаях невозможно рассмотрение лишь радиальной составляющей дисперсионного соотношения,
    поскольку учёт зависимости $\varepsilon(\varphi)$ будет всегда давать более низкий порог Оже-рекомбинации.

    Более того, наличие подобных боковых минимумов превращает подобную структуру в квазинепрямозонную, что может радикально снижать возможность излучательной 
    рекомбинации из этих точек k-пространства. 

    Также в качестве осложняющего обстоятельства нельзя не упомянуть существенное различие концентрации кадмия в квантовых ямах в плоскости структуры,
    что, с одной стороны, обуславливает возможность изучения образцов с одинаковой структурой, с другой - плохо влияет на воспроизводимость измерений,
    а также осложняет сравнение полученных результатов с теорией.

    Возможно перспективным будет являться создание структур с легированием, что позволит искусственно повысить одного из типов носителей заряда и реализовать 
    ситуацию в которой, вопреки дисперсионному соотношению будет превалировать тот или иной механизм Оже-рекомбинации. Это может быть полезным как с точки зрения 
    фундаментальных исследований, так и с точки зрения простоты оптимизации структур (можно будет заботиться о темпе рекомбинации всего по одному механизму). 
    Более того это может интересно в перспективе с позиции создания быстрых полупроводниковых детектеров, работающих в этой же области электромагнитного спектра 
    (в данном случае оже процессы будут играть уже положительную роль).
    \newpage
\end{document}