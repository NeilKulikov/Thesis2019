\documentclass[../main.tex]{subfiles}

\begin{document}
    \chapter{Виды рекомбинации}
    \section{Общие сведения }
    Как известно в полупроводниках существует несколько видов рекомбинации:
    излучательная, рекомбинация Шокли-Рида-Холла и оже-рекомбинация. 
    При этом последние два процесса происходят без излучения ЭМ волн и приводят
    к паразитному уменьшению инверсии населённостей, пользуясь терминами 
    лазерной техники. Излучательный же процесс является целевым при таком подходе.
    \vspace{0.75cm}
    \begin{figure}[h]
        \begin{minipage}[h]{0.31\linewidth}
            \begin{center}
            \begin{tikzpicture}
                \begin{scope}[very thick,decoration={
                  markings,
                  mark=at position 0.5 with {\arrow{>}}}
                  ]
                \draw[domain=-1.3:1.3,smooth,variable=\x] plot ({\x},{0.5 * \x * \x + 0.75});
                \draw[domain=-2:2,smooth,variable=\x] plot ({\x},{-0.25 * \x * \x - 0.75});
                \draw[color=black, fill=blue] (0., 0.75) circle (.1);
                \draw[postaction={decorate}, blue, thick] (0., 0.75) -- (0., -0.75); 
                \draw[color=black, fill=red] (0., -0.75) circle (.1);
                \draw[->, snake it] (0.1, 0.) -- (1.1, 0.);
                \end{scope}
              \end{tikzpicture}
              \vspace{0.75cm}
              \textbf{Излучательная}
            \end{center}
        \end{minipage}
        \hfill
        \begin{minipage}[h]{0.31\linewidth}
            \begin{center}
                \begin{tikzpicture}
                    \begin{scope}[very thick,decoration={
                      markings,
                      mark=at position 0.5 with {\arrow{>}}}
                      ]
                    \draw[domain=-1.3:1.3,smooth,variable=\x] plot ({\x},{0.5 * \x * \x + 0.75});
                    \draw[domain=-2:2,smooth,variable=\x] plot ({\x},{-0.25 * \x * \x - 0.75});
                    \draw (-0.5, 0.) -- (0.5, 0.);
                    \draw[color=black, fill=blue] (0., 0.75) circle (.1);
                    \draw[postaction={decorate}, blue, thick] (0., 0.75) -- (0., 0.); 
                    \draw[color=black, fill=red] (0., -0.75) circle (.1);
                    \draw[postaction={decorate}, red, thick] (0., -0.75) -- (0., 0.); 
                    \end{scope}
                  \end{tikzpicture}
                  \vspace{0.75cm}
                  \textbf{Рекомбинация ШРХ}
                \end{center}
            \end{minipage}
            \hfill
            \begin{minipage}[h]{0.31\linewidth}
                \begin{center}
                    \begin{tikzpicture}
                        \begin{scope}[very thick,decoration={
                          markings,
                          mark=at position 0.5 with {\arrow{>}}}
                          ]
                        \draw[domain=-2:2,smooth,variable=\x] plot ({\x},{0.5 * \x * \x + 0.25});
                        \draw[domain=-2:2,smooth,variable=\x] plot ({\x},{-0.25 * \x * \x - 0.25});
                        \draw[color=black, fill=blue] (0.289, 0.391) circle (.1);
                        \draw[color=black, fill=blue] (0.289, 0.191) circle (.1);
                        \draw[postaction={decorate}, red, thick] (-0.577, -0.333) -- (0.289, 0.291);
                        \draw[postaction={decorate}, blue, thick] (0.289, 0.291) -- (1.15, 0.918); 
                        \draw[color=black, fill=red] (-0.577, -0.333) circle (.1);
                        \draw[color=black, fill=white] (1.15, 0.918) circle (.1);
                        \end{scope}
                      \end{tikzpicture}
                      \vspace{0.75cm}
                      \textbf{Оже-рекомбинация}
                    \end{center}
                \end{minipage}
    \end{figure}

    Излучательная рекомбинация в целом делится на вынужденную и спонтанную, имеющие
    отличные коэффициенты Эйнштейна. Однако в силу требования наличия равновесия 
    оказывается, что эти коэффициенты 
    ($\tau_{sp} = 1 / A_{21},~\tau_{st} = 1/B_{21} \rho(\nu)$) связаны:
    \begin{equation*}
        \frac{A_{21}}{B_{21}} = \frac{8\pi h \nu^3}{c^3};
    \end{equation*}

    А сам коэффициент Эйнштейна для спонтанного излучения 
    ($E_g$ - ширина запрещённой зоны; $m_0,~m_h,~m_c$ - масса свободного электрона,
    эффективная масса электрона и дырки; $\Delta$ - величина спин-орбитального 
    расщепления; $P^2$ - коэффициент Кейна) \cite{Asryan:IOP:2005}: 

    \begin{equation*}
        \begin{array}{l}
            A_{21} \propto \frac{(\varepsilon_g + \varepsilon_n + \varepsilon_p) P^2}
                {\sqrt{(m_c + m_h)T}} n_n n_p \sim n_n n_p \sqrt{\frac{\varepsilon_g}{T}} ;\\
            P^2 \propto \frac{m_c + m_0}{m_c m_0} \frac{\varepsilon_g(\varepsilon_g + \Delta)}
                {\varepsilon_g + 2 \Delta / 3} \propto \frac{m_0}{m_0 m_c} \frac{\varepsilon_g \Delta}{\Delta} \sim \text{const};
        \end{array}
    \end{equation*} 

    Откуда следует вывод о том, что вероятность излучательной рекомбинации растёт
    с ростом ширины запрещённой зоны. В нашем же случаеособый интерес представляют
    образцы с малым $E_g$, а потому излучательная рекомбинация зачастую оказывается
    подавленной.

    Однако не менее важным оказывается влияние безызлучательных процессов - они снижают
    инверсию населённостей, и тем самым конкурируют с излучательными процессами. При больших 
    длинах волн оказывается, что в силу низких темпов излучательной рекомбинации именно они
    определяют время жизни носителей и тем самым величину инверсии населённостей.

    Исторически первым открытвм процессом безызлучательной рекомбинации был процесс 
    Шокли-Рида-Холла. Он протекает с участием примеси, а потому может не учитываться 
    в случае чистых полупроводников.

    В то же время существует ещё один процесс, который оказывается связан с видом дисперсионных
    соотношений для электронов и дырок. Оказывается возможным процесс в который вступает
    3-и частицы: 2-е одного знака, 1-а - иного. В ходе такого процесса две частицы разных знаков
    взаимоуничтожаются, отдавая свою энерги и импульс третьей. Таким образом происходит не только
    уничтожение носителей заряда, но и существенный разогрев оставшихся. 

    \section{Порог оже-рекомбинации}

        При этом этот процесс оказывается пороговым, иначе говоря существует минимальная "кинетическая"
    энергия трёх частиц, при которой становятся возможными такие переходы $\varepsilon_{th}$. Наличие этого 
    порога объясняется требованием выполнения законов сохранения импульса и энергии. 
    
    \begin{figure}[h!!!]
        \begin{minipage}[h]{0.66\linewidth}
        В самом деле,
        если мы возьмём все три частицы в гамма точке, то есть нулевой
        начальной "кинетической" энергией мы не сможем добиться выполнения закона сохранения энергии - 
        конечной частице просто негде будет находиться после прохождения такого процесса.
        А, если процесс с нулевой энергией невозможен, значит существует порог.
        \end{minipage}
        \hfill
        \begin{minipage}[h]{0.33\linewidth}
            \begin{center}
                \begin{tikzpicture}
                    \begin{scope}[very thick,decoration={
                        markings,
                        mark=at position 0.5 with {\arrow{>}}}
                        ]
                      \draw[domain=-2:2,smooth,variable=\x] plot ({\x},{0.5 * \x * \x + 0.25});
                      \draw[domain=-2:2,smooth,variable=\x] plot ({\x},{-0.25 * \x * \x - 0.25});
                      \draw[color=black, fill=blue] (0.1, 0.25) circle (.1);
                      \draw[color=black, fill=blue] (-0.1, 0.25) circle (.1);
                      \draw[postaction={decorate}, red, thick] (0.0, -0.25) -- (0.0, 0.25);
                      \draw[postaction={decorate}, blue, thick] (0.0, 0.25) -- (0.0, 0.75); 
                      \draw[color=black, fill=red] (0.0, -0.25) circle (.1);
                      \draw[color=black, fill=white] (0.0, 0.75) circle (.1);
                      \end{scope}
                    \end{tikzpicture}
            \end{center}
        \end{minipage}
    \end{figure}

    Его можно проиллюстрировать на примере параболических изотропных дисперсионных 
    соотношений для электронов и дырок \cite{AbakumovYassievich1997}. Оказывается,
    что в таком случае пороговая энергия и импульсы начальных частиц (два электрона
    в таком случаек должны находиться в одном состоянии) равны:

        \begin{equation}
            \begin{array}{l}
                E_{th} = E_g \frac{m_c}{m_h ( 1 + m_c / m_h)};\\
                p_{h} = p_{c} \frac{m_h}{m_c} = \sqrt{\frac{2m_c E_g}{(1+ 2m_c/m_h)(1+ m_c / m_h)}};
            \end{array}
        \end{equation}

    Как можно видеть, сама пороговая энергия пропорциональна ширине запрещённой зоны,
    откуда следует, что длинноволновые  лазеры будут испытывать большее влияние этого 
    типа рекомбинации.

    Важно отметить, что не меньшую роль играет соотношение эффективных
    масс. Очевидно, что чем больше отношение $\mu = m_c / m_h$, тем больше будет порог.
\end{document}