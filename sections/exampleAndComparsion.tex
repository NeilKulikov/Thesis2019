\begin{document}
    В ходе большого числа экспериментов по описанной выше методике было установлено, что пороговая энергия 
    Оже-рекомбинации структуры рассчитаная в изотропном приближении неплохо кореллирует с температурой затухания 
    вынужденного излучения в образце . При этом существует примерное равенство $T_{max} \approx E_{th} / 2$ 
    \cite{TmaxEthEq}. Это может быть объяснено, если учесть крайне высокие темпы термолизации носителей, что приводит
    к распределению, близкому к Больцмановскому.

    Для демонстрации будет использован пример из статьи \cite{Utochkin}. Рассматривался образец, содержащий 10 QW
    состава Hg${}_{0.903}$Cd${}_{0.097}$Te/Cd${}_0.7$Hg${}_{0.3}$Te и толщиной 7.4 $nm$, что соответствует ширине 
    запрещённой зоны около 90 $meV$ при $T = 18 ~K$. Структура не была намеренно легирована; остаточная концентрация 
    носителей p-типа, полученная на основе холловских измерений, составляла порядка единиц $10^10~cm^{-2}$, а типичная 
    плотность дислокаций $\sim 10^6~cm^{-2}$. Дизайн структуры ориентирован на эффективную локализацию света вблизи КЯ, для 
    чего массив КЯ был выращен в волноводном слое толщиной в единицы мкм. Выбранное направление роста (013) 
    препятствует использованию сколотых граней кристалла в качестве зеркал резонатора, т.к. плоскости сколов образуют
    острый угол с плоскостью QW.

    Результаты расчётов спектра электронов и дырок для двух случаев приведены на рис. 1. В первом случае (рис. 1 а) расчёт
    проведён для экспериментально исследованной Cd0.1Hg0.9Te/Cd0.65Hg0.35Te КЯ толщиной 8.7 нм, во втором случае (рис. 1 б)
    расчёт проведён для HgTe/Cd0.65Hg0.35Te КЯ толщиной 4.2 nm.

    Из рисунка 1a видно, что для первого случая в валентных подзонах имеются дополнительные максимумы, располагающиеся ниже 
    потолка валентной зоны на 7 meV. В HgTe яме, окруженной Cd0.65Hg0.35Te, эти экстремумы практически отсутствуют (рис. 1 б). 
    Как будет показано ниже, вид закона дисперсии дырок будет важен при определении величины пороговой энергии оже-рекомбинации.
    В волноводной структуре с Cd0.1Hg0.9Te КЯ-ми толщиной 8.7 нм и барьерами Cd0.65Hg0.35Te при оптической накачке было получено 
    стимулированное излучение на длине волны 18 $\mu m$, которое наблюдалось в диапазоне температур от 20 до 40К. Исследуемая 
    структура была выращена методом МПЭ на полуизолирующей GaAs (013) подложке с ZnTe и CdTe буферами. Гетероструктура содержит 
    десять КЯ Cd0.1Hg0.9Te/Cd0.65Hg0.35Te, разделенных 30 nm барьерами.
    
    Отметим, что пороговая энергия в процессе CHCC определяется в основном кинетической энергией дырки. Если учесть, что кинетическая
    энергия дырок в условиях инверсии населенности определяется не только температурой, но и положением квазиуровня Ферми в валентной 
    зоне, то согласие теории и эксперимента можно считать удовлетворительным.

    Интересно сравнить пороговые энергии для структуры, описанной выше, и структур на основе квантовых ям HgTe. На рис. 2 представлена 
    зависимость пороговой энергии оже-рекомбинации (вычисленной в модели) и толщины КЯ HgTe от доли Cd в барьерах для двух температур: 
    20 К и 77 К при фиксированной энергии оптического перехода 70 meV. Из рисунка видно, что максимальная пороговая энергия (что оптимально 
    с точки зрения максимальной температуры генерации стимулированного излучения) достигает величины 30 meV при доле Cd 0.67 для температуры 
    20 К и 27 meV при доле Cd 0.62 для температуры 77 К.

    Для объяснения различия пороговых энергий в квантовых ямах HgTe и Cd0.1Hg0.9Te на рис.1 приведены начальные и конечные состояния электронов 
    и дырок, соответствующие порогу оже-рекомбинации. Из сравнения видно, что «эффективная масса» дырок для оже-процесса в HgTe квантовой яме 
    существенно меньше, чем в Cd0.1Hg0.9Te квантовой яме. Это связано с наличием в в Cd0.1Hg0.9Te квантовой яме ярко выраженного бокового 
    экстремума в верхней валентной подзоне. Хорошо известно, что увеличение эффективной массы дырок приводит к снижению пороговой энергии 
    Оже-процесса. Наличие максимума на зависимости пороговой энергии от доли кадмия в HgTe/CdxHg1-xTe обусловлено наличием минимума 
    «эффективной массы» дырок при определенной доле кадмия. Следует отметить, условность использованного здесь термина «эффективная масса» 
    для верхней валентной подзоны, поскольку закон дисперсии в ней не квадратичный и, вообще говоря, немонотонный.

    Таким образом, было продемонстрировано, что при заданной энергии межзонного перехода пороговая энергия оже-рекомбинации в структурах 
    HgTe/CdxHg1-xTe с КЯ является немонотонной функцией от доли кадмия в барьере. При оптимальной концентрации кадмия в барьерах и КЯ из HgTe 
    можно ожидать почти трехкратного повышения критической температуры стимулированного излучения по сравнению с прототипной структурой с 
    Cd0.1Hg0.9Te/Cd0.65Hg0.35Te КЯ.

    \newpage
\end{document}